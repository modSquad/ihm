% Quelques raccourcis !
\newcommand\topologie[1]{Ajouter, modifier ou supprimer #1 (configuration de la topologie de l'aéroport)}
\newcommand\selectionner[1]{Sélectionner #1 (afficher sa configuration)}
\newcommand\rectetiquette{Rectangle étiquetté}
\newcommand\transit{Visualiser le transit de bagages dans cette zone}
\newcommand\etat{Afficher l'état}

\section{u-Aéroport}
\subsection{Définition}
Objet unique représentant l'aéroport dans son ensemble. Cet objet regroupe tous les
composants, c'est la racine du modèle.

\subsection{Présentations}
\begin{enumerate}
	\item Icône (logo de l'aéroport), accessible à tous les utilisateurs
	\item Thème standard du système d'exploitation hôte de l'application (SGBag)
	\item Informations sur l'utilisateur courant
\end{enumerate}

\subsection{Actions}
\begin{enumerate}
	\item Lancer l'application (tous, le résultat dépend des droits d'accès de l'utilisateur)
	\item L'utilisateur s'authentifie.
	\item L'utilisateur peut accéder à l'environnement de travail le concernant (litiges, gestion des bagages, \ldots)
\end{enumerate}

\section{u-Avion}
\subsection{Définition}
	Moyen de transport aérien permettant de transporter les voyageurs et leurs bagages d'un
	aéroport à un autre (un avion correspond au voyage qu'il parcours selon le contexte).

\subsection{Présentations}
\begin{enumerate}
	\item Identifiant et destination du vol (texte)
\end{enumerate}

\subsection{Actions}
\begin{enumerate}
	\item Analyser l'état de charge des bagages pour un avion
	\item Sélectionner l'avion comme destination d'un bagage
\end{enumerate}

\section{u-Terminal}
\subsection{Définition}
	Bâtiment permettant le transfert des passagers entre leur moyen de transport terrestre vers les équipements permettant d'embarquer ou débarquer d'un avion.

\subsection{Présentations}
\begin{enumerate}
	\item Nom du terminal (texte)
\end{enumerate}

\subsection{Actions}
\begin{enumerate}
	\item \selectionner{le terminal}
	\item \topologie{un terminal}
\end{enumerate}

\section{u-Chemin de connexion entre les halls}
\subsection{Définition}
	Couloir permettant à une personne de passer d'un hall à l'autre.

\subsection{Présentations}
\begin{enumerate}
	\item Identifiant et nom (texte)
\end{enumerate}

\subsection{Actions}
\begin{enumerate}
	\item \selectionner{le chemin de connexion}
	\item \topologie{un chemin de connexion}
\end{enumerate}

\section{u-Hall}
\subsection{Définition}
	Espace permettant l'accueil des visiteurs de l'aéroport ou de l'un de ses secteurs.
\subsection{Présentations}
\begin{enumerate}
	\item Identifiant et nom (texte)
	\item cartographie du terminal (composition d'éléments atomiques)
\end{enumerate}

\subsection{Actions}
\begin{enumerate}
	\item \selectionner{le hall}
	\item \topologie{un hall}
\end{enumerate}

\section{u-Guichet d'enregistrement}
\subsection{Définition}
	Guichet permettant à un voyageur de se déclarer avant l'accès au terminal.

\subsection{Présentations}
\begin{enumerate}
	\item Identifiant et nom (texte)
	\item Icône (rectangle rayé dans le sens de circulation des bagages)
\end{enumerate}

\subsection{Actions}
\begin{enumerate}
	\item \selectionner{le guichet d'enregistrement}
	\item \topologie{un guichet d'enregistrement}
\end{enumerate}

\section{u-Bagage}
\subsection{Définition}
	Objets (regroupés) qu'un voyageur transporte avec lui. Les bagages peuvent être des bagages à main (qu'il conserve pendant
	le voyage) ou des bagages transportés dans la soute de l'avion.

\subsection{Présentations}
\begin{enumerate}
	\item Identifiant du bagage (d'après la puce RFID passive)
\end{enumerate}

\subsection{Actions}
\begin{enumerate}
	\item Enregistrer le bagage
	\item Modifier l'état du bagage manuellement
	\item Dérouter le bagage (le chariot)
\end{enumerate}

\section{u-Chariot}
\subsection{Définition}
	Conteneur transportant les bagages d'un voyageur sur le circuit de rails.
\subsection{Présentations}
\begin{enumerate}
	\item Icône (Cercle plein)
	\item Identifiant du chariot
\end{enumerate}

\subsection{Actions}
\begin{enumerate}
	\item Dérouter un chariot de son circuit
	\item Dérouter un chariot (vide) vers la zone de chargement des batteries
	\item \etat
	\item Création, modification ou suppression de l'objet (configuration)
\end{enumerate}

\section{u-Rail}
\subsection{Définition}
	Composant du circuit empreinte par un train acheminant les bagages de la zone d'embarquement vers un avion.

\subsection{Présentations}
\begin{enumerate}
	\item Identifiant (texte)
\end{enumerate}

\subsection{Actions}
\begin{enumerate}
	\item Analyse du trafic sur les rails
	\item \etat 
	\item \topologie{un rail}
\end{enumerate}

\section{u-Zone de contrôle de sécurité Automatique}
\subsection{Définition}
	Zone par laquelle passent les chariots de bagages pour qu'un système automatisé contrôle le contenu des bagages pour y
	détecter d'éventuels contenus à risque.

\subsection{Présentations}
\begin{enumerate}
	\item \rectetiquette
\end{enumerate}

\subsection{Actions}
\begin{enumerate}
	\item Visualiser l'état de la zone de contrôle
	\item \topologie{une zone de contrôle}
	\item \transit
\end{enumerate}

\section{u-Portique}
\subsection{Définition}
	Système automatisé permettant la détection de métaux par lequel un voyageur doit passer avant d'embarquer.

\subsection{Présentations}
\begin{enumerate}
	\item Icône (trait épais)
	\item Identifiant (texte)
\end{enumerate}

\subsection{Actions}
\begin{enumerate}
	\item consulter l'objet
	\item \etat
\end{enumerate}

\section{u-Zone de contrôle de sécurité Manuelle}
\subsection{Définition}
	Zone vers laquelle des bagages ayant échoué au contrôle de sécurité automatique sont acheminés pour qu'un contrôle manuel
	soit effectué par un contrôleur.

\subsection{Présentations}
\begin{enumerate}
	\item \rectetiquette
\end{enumerate}

\subsection{Actions}
\begin{enumerate}
	\item \selectionner{la zone de contrôle de sécurité Manuelle}
	\item \topologie{une zone de contrôle de sécurité Manuelle}
	\item \transit
\end{enumerate}

\section{u-Voie de garage}
\subsection{Définition}
	Voie sans issue vers laquelle un chariot est dérouté du circuit pour qu'il subisse un traitement (par exemple, un chariot
	de bagages ayant échoué au contrôle de sécurité automatique).

\subsection{Présentations}
\begin{enumerate}
	\item Semblable à un chemin de roulement
\end{enumerate}

\subsection{Actions}
\begin{enumerate}
	\item \selectionner{la voie de garage}
	\item \topologie{une voie de garage}
	\item Désactiver la voie
	\item Activer la voie
	\item \etat
\end{enumerate}

\section{u-Zone de chargement des batteries}
\subsection{Définition}
	Zone dans laquelle les chariots rechargent leurs batteries d'alimentation électrique.

\subsection{Présentations}
\begin{enumerate}
	\item \rectetiquette
\end{enumerate}

\subsection{Actions}
\begin{enumerate}
	\item \selectionner{la zone de chargement des batteries}
	\item \topologie{une zone de chargement des batteries}
	\item Afficher la liste des chariots en charge
	\item \transit
\end{enumerate}

\section{u-Zone embarquement}
\subsection{Définition}
	Zone dans laquelle les bagages sont chargés dans des wagonnets en prévision de leur acheminement vers l'avion.

\subsection{Présentations}
\begin{enumerate}
	\item \rectetiquette
\end{enumerate}

\subsection{Actions}
\begin{enumerate}
	\item \selectionner{la zone d'embarquement}
	\item \topologie{une zone d'embarquement}
	\item Afficher les containers à être transportés
	\item \transit
\end{enumerate}

\section{u-Tapis roulant}
\subsection{Définition}
	Dispositif automatique permettant le déplacement de bagages d'un point à un autre et pouvant effectuer la pesée d'un
	bagage.

\subsection{Présentations}
\begin{enumerate}
	\item Rectangle rayé horizontalement, fléché dans le sens de
	circulation.
\end{enumerate}

\subsection{Actions}
\begin{enumerate}
	\item \selectionner{le tapis roulant}
	\item \topologie{un tapis roulant}
	\item Visualiser le trafic du tapis roulant
	\item Désactiver le tapis roulant
	\item Activer le tapis roulant
	\item \etat
\end{enumerate}

\section{u-Train}
\subsection{Définition}
	Moyen de transport utilisé pour déplacer les containers de bagages de la zone d'embarquement à leur avion.

\subsection{Présentations}
\begin{enumerate}
	\item Identifiant (texte)
\end{enumerate}

\subsection{Actions}
\begin{enumerate}
	\item Débuter le chargement
	\item Terminer le chargement
	\item \selectionner{le train}
	\item Création, modification et suppression d'un train (configuration)
	\item Mise à jour de la composition d'un train
	\item \etat
\end{enumerate}

\section{u-Wagonnets}
\subsection{Définition}
	Composant du train dans lequel sont chargés les containers de bagages.

\subsection{Présentations}
\begin{enumerate}
	\item Identifiant du wagonnet (texte)
\end{enumerate}

\subsection{Actions}
\begin{enumerate}
	\item Enregistrer la présence (ou non) d'un container
	\item Création, modification et suppression d'un wagonnet (configuration)
	\item \etat
\end{enumerate}

\section{u-Containers}
\subsection{Définition}
	Conteneur de bagages à grand volume dans lequel les bagages sont chargés pour leur déplacement par le train.

\subsection{Présentations}
\begin{enumerate}
	\item Identifiant (texte)
\end{enumerate}

\subsection{Actions}
\begin{enumerate}
	\item Associer un container à un wagonnet
	\item Charger le container dans l'avion
	\item Décharger le container de l'avion
	\item Remplir le container
	\item Vider le container
	\item Création, modification et suppression d'un container (configuration)
	\item \etat
\end{enumerate}

\section{u-Plateau élévateur}
\subsection{Définition}
	Dispositif automatique permettant d'élever un ensemble de bagages.

\subsection{Présentations}
\begin{enumerate}
	\item Icône : double rectangle plein
	\item Identifiant de l'objet (texte)
\end{enumerate}

\subsection{Actions}
\begin{enumerate}
	\item \selectionner{le plateau élévateur}
	\item \topologie{un plateau élévateur}
	\item \etat
	\item Désactiver le plateau
	\item Activer le plateau
\end{enumerate}

\section{u-Zone déchargement}
\subsection{Définition}
	Zone dans laquelle les bagages arrivés à leurs destinations sont déchargés des containers vers le circuits de rails pour
	qu'ils soient acheminés vers la zone de retrait des bagages.

\subsection{Présentations}
\begin{enumerate}
	\item \rectetiquette
\end{enumerate}

\subsection{Actions}
\begin{enumerate}
	\item \selectionner{la zone de déchargement}
	\item \topologie{une zone de déchargement}
	\item \transit
\end{enumerate}

\section{u-Tobogan}
\subsection{Définition}
	Dispositif permettant de faire glisser des bagages du circuits de rails vers un container.

\subsection{Présentations}
\begin{enumerate}
	\item Idenfiant de l'objet (texte)
	\item Icône : triangle isocèle sans base et sans sommet, repli par une flêche vers la base
\end{enumerate}

\subsection{Actions}
\begin{enumerate}
	\item \selectionner{le tobogan}
	\item \topologie{tobogan}
	\item \etat
	\item Désactiver la circulation par le tobogan
	\item Activer la circulation par le tobogan
\end{enumerate}


\section{u-Zone retrait des bagages}
\subsection{Définition}
	Zone dans laquelle est situé un carrousel permettant au voyageurs ayant débarqués de récupérer les bagages transportés
	en soute.

\subsection{Présentations}
\begin{enumerate}
	\item Identifiant de l'objet (texte)
	\item \rectetiquette
\end{enumerate}

\subsection{Actions}
\begin{enumerate}
	\item \selectionner{la zone de retrait des bagages}
	\item \topologie{une zone de bagages}
	\item \transit
\end{enumerate}

\section{u-Carrousel}
\subsection{Définition}
	Tapis roulant circulaire sur lequel sont déposés les bagages transportés en soute prêts à être récupérés par les voyageurs
	ayant débarqué.

\subsection{Présentations}
\begin{enumerate}
	\item Icône : tube circulaire (rayé horizontalement)
	\item Identifiant (texte)
\end{enumerate}

\subsection{Actions}
\begin{enumerate}
	\item Création, modification ou suppression d'un carrousel (configuration)
	\item \etat
	\item Désactiver le carrousel
	\item Activer le carrousel
\end{enumerate}

\section{u-Zone de maintenance}
\subsection{Définition}
	Zone dans laquelle des opérations de maintenance sont effectués sur les composants du circuit d'acheminement des bagages
	(chariots, rails, ...).

\subsection{Présentations}
\begin{enumerate}
	\item \rectetiquette
	\item Identifiant (texte)
\end{enumerate}

\subsection{Actions}
\begin{enumerate}
	\item Visualiser les objets en zone de maintenance
	\item Consulter l'historique des objets passés en zone de maintenance
\end{enumerate}

\section{u-Aiguillage}
\subsection{Définition}
	Composant du circuit effectuant la jonction entre deux voies.

\subsection{Présentations}
\begin{enumerate}
	\item Séparation de tuyau
	\item Identifiant (texte)
\end{enumerate}

\subsection{Actions}
\begin{enumerate}
	\item Création, modification ou suppression d'un aiguillage (configuration)
	\item \etat
	\item \transit
	\item Changer l'état de l'aiguillage
	\item Désactiver la circulation par cet aiguillage
	\item Activer la circulation par cet aiguillage
\end{enumerate}

\section{u-Chemin de roulement}
\subsection{Définition}
	circuit empreinte par les chariots de bagages, effectuant la liaison entre un hall, les différents points de contrôle
	et les zones de chargement/déchargement des bagages.

\subsection{Présentations}
\begin{enumerate}
	\item Icône : trait plein épais (ou "tuyau"), contient des flèches indiquant le sens de circulation
	\item Identifiant (texte)
\end{enumerate}

\subsection{Actions}
\begin{enumerate}
	\item Création, modification ou suppression d'un chemin de roulement (configuration)
	\item \etat
	\item \transit
	\item Désactiver la circulation par ce chemin
	\item Activer la circulation par ce chemin
\end{enumerate}

\section{u-Événement}
\subsection{Définition}
	Fait remarquable dans le processus d'acheminement des bagages (comme l'enregistrement d'un nouveau bagage, l'arrivée
	d'un chariot à sa destination, sa déviation ou encore un événement accidentel).

\subsection{Présentations}
\begin{enumerate}
	\item Texte descriptif
	\item Item dans une liste
	\item Icône : symbole normalisé «attention» (point d'exclamation dans une forme à bordure rouge)
\end{enumerate}

\subsection{Actions}
\begin{enumerate}
	\item Visualiser l'événement
	\item Visualiser les bagages associés au contexte de l'événement
\end{enumerate}

\section{u-Données d'exploitation}
\subsection{Définition}
	Ensemble de données représentant une trace du pilotage du processus d'acheminement des bagages enregistrée par un
	système automatisé.

\subsection{Présentations}
\begin{enumerate}
	\item Item dans une liste
	\item Identifiant de l'enregistrement
	\item Icône : fichier "standard"
\end{enumerate}

\subsection{Actions}
\begin{enumerate}
	\item Visualiser les données d'exploitation
	\item Purger les données
	\item Archiver les données
\end{enumerate}
