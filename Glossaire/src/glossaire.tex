\plex{Anomalie grave}{}
\plex{Bagage perdu}{C'est un bagage dont on arrive pas à retrouver la trace dans le système. Voir Voyageur Lésé}
\plex{Bagage}{Objet appartenant à un voyageur lui permettant de transporter ses affaires lorsqu'il prend l'avion. Acheminé vers et depuis l'avion par des tapis roulants. Par exemple : valise, mal, sac à dos, etc.} \plex{Chariot}{}
\plex{Chariot}{Appareil placé en fin de tapis roulant lors du chargement, permettant de récupérer les bagages déposés par les voyageurs.}
\plex{Circuit d'un bagage}{C'est le chemin que prend le bagage }
\plex{Client}{Du point de vue du système, un client est une personne qui ne fait pas parti du personnel de l'aéroport, et qui utilise le service de l'aéroport.}
\plex{Commande manuelles}{À mettre en regard avec le \emph{pilotage automatique} du système, le mode commande manuelle peut s'avérer indispensable quand le système rencontre une anomalie de fonctionnement. Les commandes manuelles sont contrôlées, et inhibées si le système détecte une situation dangereuse.}
\plex{Compte utilisateur}{Repéré par un numéro d'identification associé à un mot de passe, ce compte permet à un voyageur lésé de consulter l'état de résolution de son problème.}
\plex{DDF}{Définition du domaine fonctionnel.}
\plex{DPU}{Description des profils utilisateurs.}
\plex{DTU}{Définition des tâches utilisateurs.}
\plex{Dossier ou dossier d'un client}{Ensemble des information connu du client par le système. Le dossier d'un client est entre autre consultable sur les PDA des personnels de l'aéroport.}
\plex{Débarquement}{Phase qui débute lorsqu'un avion arrive, et qui se termine lorsque le processus de déchargement est terminé.}
\plex{Embarquement}{
    \begin{enumerate}
	\item Période temporelle pendant laquelle les voyageurs sont sensés
	    embarquer. Les bagages sont chargés pendant cette période.
	\item Endroit où les voyageurs embarquent dans l'avion (Zone d').
    \end{enumerate}
}
\plex{Étiquette}{Munie d'une puce RFID passive et accrochée à un bagage, elle comporte la destination du bagage et un code barre. Elle est imprimée par l'opérateur de guichet lorsqu'un passager enregistre ses bagages.}
\plex{GPU}{Graphe des profils utilisateur.}
\plex{Journaux de l'application}{Aussi et souvent appelé \emph{log file}, fichier texte, le plus souvent daté, qui permet de rendre compte du fonctionnement de l'application. Permet nettement de savoir la cause de certaines erreurs qui se produisent éventuellement dans le système. Un contrôle de niveau permet de réglé le niveau de détails souhaités, en fonction du contexte : phase de test, mise en production, etc.}
\plex{MU}{Modèle des utilisateurs.}
\plex{PDA}{De l'anglais «~Personal Digital Assistant~», c'est un dispositif électronique capable de gérer les informations personnelles de l'utilisateur, et éventuellement de communiquer.}
\plex{Pilotage automatique}{Système qui permet de connaitre l'état du système à
chaque instant. Par un système de capteurs placés à des points de contrôles, les
bagages sont routés à travers le système. Le pilotage automatique peut être
contrôlé ponctuellement par des commandes manuelles.}
\plex{Profil utilisateurs}{Catégorisation des utilisateurs selon l'utilisation qu'ils vont faire du système.}
\plex{SGBag}{Le nom de l'application de gestion de bagage pour aéroport dont l'interface graphique est spécifiée par l'ensemble des documents fournis avec ce glossaire}
\plex{SI}{Pour Système Informatique, décrit l'ensemble des équipements matériels informatiques et logiciels.}
\plex{Simulation}{Mode de fonctionnement du l'application SGBag, pour tester le système, où les problématiques matérielles sont remplacées par du logiciel, pour tester certains cas.}
\plex{Statistiques}{Données chiffrées, par exemple des type de panne par heure de la journée, qui sont collectés au sein du système par l'étude des journaux de l'application.}
\plex{Système}{Le système est l'ensemble de la structure mise en place pour le bon déroulement des opération. Il inclue du personnel, du matériel, et du logiciel.}
\plex{TTU-DF}{Table des traitements utilisateurs.}
\plex{Tapis roulant}{Dispositif mécanique qui permet de déplacer des bagages dans un aéroport, afin de les acheminer aux voyageurs, qui attendent dans le hall pour les récupérer. Il peuvent aussi être utilisés pour pour que les voyageurs déposent leurs bagages. Il sont alors munis de balances pour peser les bagages.}
\plex{Train bagages}{Dispositif consistant en un chainage de wagonnets, conduit par un humain, qui déplace les bagages, placés dans des containers, de la sortie du toboggan jusqu'à l'avion.}
\plex{Visualisation}{Vu spécifique du système, en fonction des besoins d'un opérateur : par zone, globale, caméra sur une zone, etc.}
\plex{Voyageur lésé}{C'est un voyageur dont on a perdu le(s) bagage(s).}
\plex{Wagonnet}{Petit wagon utilisé pour transporter les bagages des voyageurs sur le tarmac de l'aéroport. Les wagonnets sont reliés pour former un train.}
\plex{Zone ou bien zone d'opération}{Endroit délimité qui dispose de son propre bouton d'arrêt d'urgence. Les zones sont de tailles à peu près équivalentes.}
