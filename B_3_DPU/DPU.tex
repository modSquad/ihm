\documentclass{article}
\usepackage[utf8]{inputenc}
\usepackage[francais]{babel}
\begin{document}


\section{DPU : Description des profils utilisateurs}

\subsection{Membres du service bagages}
\begin{description}
\item[Responsable du service] De formation Bac+5 (école d'ingénieur
généraliste), il est expérimenté et possède donc une excellente connaissance du
métier et maîtrise tout le processus de gestion des bagages. Son expérience avec
l'outil informatique est variable. Il est amené à prendre rapidement des
décisions et donner des directives au autres membres du service. Petit groupe
homogène : les responsables sont complémentaires.
\item[Responsable technique] Ingénieur (en informatique), il maîtrise le domaine
métier (formé par l'entreprise) et l'informatique. Il sait contrôler et
anticiper les problèmes de l'infrastructure logicielle du service. Il doit
également pouvoir prendre des décisions d'urgence et communiquer avec les autres
membres du service. Le groupe est restreint et homogène.
\item[Superviseur] De formation Bac+2 à Bac+3 ayant une bonne connaissance du
domaine métier. Il maîtrise les outils informatiques classiques : il se concentre plus
sur les aspects fonctionnels du système que les outils à mettre en place. Il
doit réagir rapidement aux imprévus. Les superviseurs sont nombreux et peuvent
avoir des profils variés.
\item[Conducteur] Titulaire d'un diplôme de technicien, le conducteur maîtrise
son outil, a une connaissance du domaine limitée à son secteur d'activité. Son
expertise en informatique est variable, mais doit être considérée comme faible.
Le groupe des conducteurs est nombreux et héterogène.
\end{description}

\subsection{Membres des services transverses}
\begin{description}
\item[Technicien du système d'informations] Titulaire d'un diplôme de technicien
en informatique (DUT, BTS, Licence, \ldots) il maîtrise l'informatique et
connait l'infrastructure mise en place dans l'aéroport. Sa connaissance du
domaine est variable et dépend généralement de son expérience dans l'entreprise.
Les techniciens du SI sont nombreux et de profils variés. Ils ne sont pas tous
capables d'intervenir sur les mêmes secteurs du SI de l'aéroport. Groupe
nombreux.
\item[Technicien de maintenance] Titulaire d'un diplôme de technicien orienté
mécanique (DUT GMP, BTS, \ldots), il a un niveau moyen en informatique (par sa
formation) et connait l'infrastructure et la machinerie de l'aéroport, mais sa
connaissance du domaine (contraintes fonctionnelles, etc) varie selon son
expérience dans l'entreprise. Groupe nombreux. 
\item[Contrôleur] Le contrôleur est formé aux contraintes du domaine mais n'a
pas de profil clairement identifiable (niveau de qualification variable). Les
contrôleurs sont nombreux et le groupe est très hétérogène.
\end{description}

\subsection{Personnel d'Accueil}
\begin{description}
\item[Opérateur de gichet] Titulaire d'une formation Bac+2 à Bac+3, l'opérateur
de guichet maîtrise généralement peu le domaine métier et l'informatique. Il est
formé aux outils mis en place par l'entreprise.
\item[Préposé aux réclamations] Titulaire d'une formation Bac+2 à Bac+3, sa
maîtrise du domaine est variable mais souvent peu pointue. Il n'est pas initié à
l'informatique, mais est formé aux outils de l'entreprise.
\end{description}

\subsection{Public}
\begin{description}
\item[Voyageur lésé] Le voyageur lesé est un voyageur ayant perdu un ou
plusieurs bagages. Son profil ne peut être clairement établi.
\end{description}

\end{document}
