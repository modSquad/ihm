\newcommand\fatal{(Erreur fatale)}
\newcommand\warning{(message d'avertissement)}
\newcommand\event{u-Événement}
\newcommand\syslog{u-Données d'exploitation}
\newcommand\urt{Responsable technique}
\newcommand\us{Superviseur}
\newcommand\allobjs{u-Aéroport, u-Avion, u-Terminal, u-Hall, u-Chemin de connexion entre les halls, u-Gichet d'enregistrement,
u-Bagage, u-Chariot, u-Rail, u-Zone de contrôle de sécurité automatique, u-Zone de contrôle de sécurité
manuelle, u-Portique, u-Voie de garage, u-Zone de chargement des batteries, u-Zone embarquement, u-Tapis roulant,
u-Train, u-Wagonnet, u-Container, u-Plateau élévateur, u-Zone de déchargement, u-Tobogan, u-Zone de retrait des bagages,
u-Carrousel, u-Zone de maintenance, u-Aiguillage, u-Chemin de roulement, u-Événement, u-Données d'exploitation}
\newcommand\circobjs{u-Zone*, u-Bagage, u-Chariot, u-Voie de garage, u-Tapis roulant, u-Train, u-Rail, u-Wagonnet, u-Container,
u-Plateau Élévateur, u-Tobogan, u-Carrousel, u-Aiguillage, u-Chemin de roulement}
\newcommand\autobjs{u-Guichet d'enregistrement, u-Chariot, u-Portique, u-Tapis roulant, u-Train, u-Plateau élévateur, u-Tobogan,
u-Carrousel, u-Chemin de roulement, \event}

\section{log}
	\textbf{Qui:} Tous les utilisateurs

	\subsection{Description}
	\begin{itemize}
		\item Permet à l'utilisateur de s'authentifier auprès du système
		\item Déclenche l'évaluation des droits de l'utilisateur
		\item Charge l'environnement de travail de l'utilisateur
	\end{itemize}

	\subsection{Entrées}
		login, password

	\subsection{Retours}
	\begin{itemize}
		\item confirmation de l'authentification
		\item identifiant de session de travail (sessid)
	\end{itemize}

	\subsection{Erreurs}
	\begin{itemize}
		\item cpoule (login, password) incorrect, inexistant dans le
		système ou ne disposant pas de droits suffisants \fatal
	\end{itemize}

	\subsection{Objet(s)}
		u-Aéroport, \event, \syslog

\section{unlog}
	\textbf{Qui:} Tous les utilisateurs

	\subsection{Description}
	\begin{itemize}
		\item Déconnecte l'utilisateur
		\item Termine la session de travail
		\item Ferme les fenêtres de l'environnement de travail
	\end{itemize}

	\subsection{Entrées}
		sessid

	\subsection{Retours}
	\begin{itemize}
		\item Message de confirmation
	\end{itemize}

	\subsection{Erreurs}
	\begin{itemize}
		\item sessid incorrect \fatal 
	\end{itemize}

	\subsection{Objet(s)}
		u-Aéroport, \event, \syslog

\section{ouvrir-inter-simulation}
	\textbf{Qui:} \urt, \us

	\subsection{Description}
	\begin{itemize}
		\item Sélectionne les éléments par défaut
		\item Charge l'environnement de simulation
	\end{itemize}

	\subsection{Entrées}
		N/A

	\subsection{Retours}
	\begin{itemize}
		\item Affichage de l'interface de simulation
	\end{itemize}

	\subsection{Erreurs}
	\begin{itemize}
		\item Droits insuffisants (fatal)
	\end{itemize}

	\subsection{Objet(s)}
		\allobjs

\section{ouvr-inter-configuration}
	\textbf{Qui:} \us, \urt

	\subsection{Description}
	\begin{itemize}
		\item Sélectionne les éléments par défaut
		\item Charge l'environnement de confiuration
	\end{itemize}

	\subsection{Entrées}
		N/A

	\subsection{Retours}
	\begin{itemize}
		\item Affichage de l'interface de configuration
	\end{itemize}

	\subsection{Erreurs}
	\begin{itemize}
		\item Droits insuffisants \fatal
	\end{itemize}

	\subsection{Objet(s)}
		\allobjs

\section{afficher-bagages}
	\textbf{Qui:} \us, \urt

	\subsection{Description}
	\begin{itemize}
		\item Affiche l'interface de visualisation des bagages pour la zone choisie
	\end{itemize}

	\subsection{Entrées}
		Zone choisie pour la visualisation

	\subsection{Retours}
	\begin{itemize}
		\item Affichage de l'interface de visualisation des bagages pour la zone choisie
		\item Liste et volume des bagages en transit dans la zone
	\end{itemize}

	\subsection{Erreurs}
	\begin{itemize}
		\item Zone incorrecte \fatal
	\end{itemize}

	\subsection{Objet(s)}
		\circobjs

\section{afficher-chariots}
	\textbf{Qui:} \us, \urt

	\subsection{Description}
	\begin{itemize}
		\item Affiche l'interface de visualisation des chariots pour la zone choisie
	\end{itemize}

	\subsection{Entrées}
		Zone choisie pour la visualisation

	\subsection{Retours}
	\begin{itemize}
		\item Affichage de l'interface de visualisation des chariots pour la zone choisie
		\item Liste les chariots en transit dans la zone
	\end{itemize}

	\subsection{Erreurs}
	\begin{itemize}
		\item Zone incorrecte \fatal
	\end{itemize}

	\subsection{Objet(s)}
		\circobjs

\section{afficher-containers}
	\textbf{Qui:} \us, \urt

	\subsection{Description}
	\begin{itemize}
		\item Affiche l'interface de visualisation des containers pour la zone choisie
	\end{itemize}

	\subsection{Entrées}
		Zone choisie pour la visualisation

	\subsection{Retours}
	\begin{itemize}
		\item Affichage de l'interface de visualisation des containers pour la zone choisie
		\item Liste les containers et leur état en transit dans la zone
	\end{itemize}

	\subsection{Erreurs}
	\begin{itemize}
		\item Zone incorrecte \fatal
	\end{itemize}

	\subsection{Objet(s)}
		\circobjs

\section{ajouter-evnmt}
	\textbf{Qui:} \urt, \us

	\subsection{Description}
	\begin{itemize}
		\item Enregistre un nouvel événement
	\end{itemize}

	\subsection{Entrées}
		pour l'événement : identifiant, date, régularité, fréquence

	\subsection{Retours}
	\begin{itemize}
		\item Nouvel événément enregistré dans le système
		\item Mise à jour des listes d'événements proches
	\end{itemize}

	\subsection{Erreurs}
	\begin{itemize}
		\item Données en entrée incorrectes \fatal
		\item Identifiant déjà attribué \fatal
	\end{itemize}

	\subsection{Objet(s)}
		u-Événement

\section{modifier-evnmt}
	\textbf{Qui:} \urt, \us

	\subsection{Description}
	\begin{itemize}
		\item Met à jour un nouvel événement
	\end{itemize}

	\subsection{Entrées}
		pour l'événement : identifiant, date, régularité, fréquence

	\subsection{Retours}
	\begin{itemize}
		\item Événément modifié enregistré dans le système
		\item Mise à jour des listes d'événements proches
	\end{itemize}

	\subsection{Erreurs}
	\begin{itemize}
		\item Données en entrée incorrectes \fatal
		\item identifiant inconnu \fatal
	\end{itemize}

	\subsection{Objet(s)}
		u-Événement

\section{supprimer-evnmnt}
	\textbf{Qui:} \urt, \us

	\subsection{Description}
	\begin{itemize}
		\item Supprime un événement
	\end{itemize}

	\subsection{Entrées}
		identifiant de l'événement

	\subsection{Retours}
	\begin{itemize}
		\item Confirmation de suppression
		\item Mise à jour des listes d'événements
	\end{itemize}

	\subsection{Erreurs}
	\begin{itemize}
		\item Mauvais identifiant \fatal
	\end{itemize}

	\subsection{Objet(s)}
		u-Événement

\section{enregistrer-sim}
	\textbf{Qui:} \urt, \us

	\subsection{Description}
	\begin{itemize}
		\item Collecte l'ensemble des paramètres de la simulation
		\item Suspend une simulation en cours
		\item Enregistre une simulation dans le système
	\end{itemize}

	\subsection{Entrées}
		Nom de la simulation

	\subsection{Retours}
	\begin{itemize}
		\item Confirmation
		\item Simulation ajoutée à la base des simulations
	\end{itemize}

	\subsection{Erreurs}
	\begin{itemize}
		\item Paramètres des objets incomplets (seuls les objets correctement décrits sont sauvegardés)
	\end{itemize}

	\subsection{Objet(s)}
		\allobjs

\section{enreg-sous-sim}
	Voir enreg-sim

	\subsection{Erreurs}
	\begin{itemize}
		\item Nom de simulation existante \fatal
	\end{itemize}

\section{charger-sim}
	\textbf{Qui:} \urt, \us

	\subsection{Description}
	\begin{itemize}
		\item Affiche la liste des simulations sauvegardées
		\item Charge la simulation sélectionnée
		\item Restaure les objets de la simulation
		\item Restaure l'état d'avancement de la simulation
	\end{itemize}

	\subsection{Entrées}
		Nom de la simulation

	\subsection{Retours}
	\begin{itemize}
		\item Affichage de l'interface de simulation
	\end{itemize}

	\subsection{Erreurs}
	\begin{itemize}
		\item Simulation inconnue \fatal
		\item Simulation incomplète \warning
	\end{itemize}

	\subsection{Objet(s)}
		\allobjs

\section{modif-chemin-chariot}
	\textbf{Qui:} \urt, \us

	\subsection{Description}
	\begin{itemize}
		\item Sélectionne un chariot
		\item Modifie la trajectoire du chariot selectionné
		\item Met à jour la simulation
	\end{itemize}

	\subsection{Entrées}
		Chariot, nouveau chemin

	\subsection{Retours}
	\begin{itemize}
		\item Confirmation
		\item Mise à jour de l'interface
		\item Mise à jour de la simulation
	\end{itemize}

	\subsection{Erreurs}
	\begin{itemize}
		\item Chemin innaccessible \warning
		\item Chariot inconnu \fatal
	\end{itemize}

	\subsection{Objet(s)}
		u-Chariot, u-Chemin de roulement

\section{modif-freq-pannes}
	\textbf{Qui:} \urt, \us 

	\subsection{Description}
	\begin{itemize}
		\item Modifie la probabilité de déclencher des pannes dans la simulation
	\end{itemize}

	\subsection{Entrées}
		valeur de la frequence

	\subsection{Retours}
	\begin{itemize}
		\item Confirmation
		\item Mise à jour le la liste des événements
		\item Mise à jour de la simulation
	\end{itemize}

	\subsection{Erreurs}
	\begin{itemize}
		\item Valeur incorrecte : la modification est annulée \warning
	\end{itemize}

	\subsection{Objet(s)}
		\circobjs, \event

\section{modif-flux-bag}
	\textbf{Qui:} \urt, \us

	\subsection{Description}
	\begin{itemize}
		\item Modifie le volume de bagages traités par la simulation
	\end{itemize}

	\subsection{Entrées}
		valeur du paramètre

	\subsection{Retours}
	\begin{itemize}
		\item Confirmation
		\item Mise à jour de la simulation
		\item Mise à jour de l'interface
	\end{itemize}

	\subsection{Erreurs}
	\begin{itemize}
		\item Valeur incorrecte : la modification est annulée \warning
	\end{itemize}

	\subsection{Objet(s)}
		u-Bagage, u-Guichet d'enregistrement

\section{selec-elmt-liste}
	\textbf{Qui:} Tous les utilisateurs (pour les éléments qui leur sont accessibles)

	\subsection{Description}
	\begin{itemize}
		\item Sélectionne un élément dans la liste des éléments
	\end{itemize}

	\subsection{Entrées}
		Identifiant de l'élément

	\subsection{Retours}
	\begin{itemize}
		\item L'élément est en surbrillance et apparait sélectionné dans la liste
	\end{itemize}

	\subsection{Erreurs}
	\begin{itemize}
		\item N/A
	\end{itemize}

	\subsection{Objet(s)}
		\allobjs

\section{selec-elmt-graph}
	\textbf{Qui:} Tous les utilisateurs (pour les éléments qui leur sont accessibles)

	\subsection{Description}
	\begin{itemize}
		\item Sélectionne un élément dans le graphe représentant le système
	\end{itemize}

	\subsection{Entrées}
		Identifiant de l'élément

	\subsection{Retours}
	\begin{itemize}
		\item L'élément est en surbrillance et apparait sélectionné sur le graphe
	\end{itemize}

	\subsection{Erreurs}
	\begin{itemize}
		\item N/A
	\end{itemize}

	\subsection{Objet(s)}
		\allobjs

\section{dem-elmt}
	\textbf{Qui:} Tous les utilisateurs (pour les éléments qui leur sont accessibles)

	\subsection{Description}
	\begin{itemize}
		\item Arrêter l'élément sélectionné
	\end{itemize}

	\subsection{Entrées}
		Élément sélectionné

	\subsection{Retours}
	\begin{itemize}
		\item Arrête l'élément sélectionné (il est désactivé)
	\end{itemize}

	\subsection{Erreurs}
	\begin{itemize}
		\item L'élément n'est pas concerné par cette action \fatal
	\end{itemize}

	\subsection{Objet(s)}
		\autobjs

\section{ajout-elmt}
	\textbf{Qui:} Tous les utilisateurs (pour les éléments qui leurs sont accessibles)

	\subsection{Description}
	\begin{itemize}
		\item Ajoute à la simulation un élément configuré
	\end{itemize}

	\subsection{Entrées}
		Type de l'élément, dimensions de l'élément, position de l'élément, liens entre les éléments existant et celui
		nouvellement créé, paramètres spécifiques au type de l'élément.

	\subsection{Retours}
	\begin{itemize}
		\item L'élément est créé
		\item Confirmation à l'utilisateur
		\item Mise à jour de la liste des éléments et du graphe
	\end{itemize}

	\subsection{Erreurs}
	\begin{itemize}
		\item Mauvaise valeur pour l'un des paramètres en entrée \fatal
	\end{itemize}

	\subsection{Objet(s)}
		\allobjs

\section{modif-elmt}
	\textbf{Qui:} \urt

	\subsection{Description}
	\begin{itemize}
		\item Met à jour un élément de la simulation
	\end{itemize}

	\subsection{Entrées}
		Type de l'élément, dimensions de l'élément, position de l'élément, liens entre les éléments existant et celui
		nouvellement créé, paramètres spécifiques au type de l'élément.

	\subsection{Retours}
	\begin{itemize}
		\item L'élément est mis à jour
		\item Confirmation à l'utilisateur
		\item Mise à jour de la liste des éléments et du graphe
	\end{itemize}

	\subsection{Erreurs}
	\begin{itemize}
		\item Mauvaise valeur pour l'un des paramètres en entrée \fatal
	\end{itemize}

	\subsection{Objet(s)}
		\allobjs

\section{suppri-elmt}
	\textbf{Qui:} \urt

	\subsection{Description}
	\begin{itemize}
		\item Supprime l'élément sélectionné
	\end{itemize}

	\subsection{Entrées}
		Identifiant de l'élément

	\subsection{Retours}
	\begin{itemize}
		\item Supprime l'élément de la simulation
		\item Confirmation à l'utilisateur
		\item Mise à jour de la liste des éléments et du graphe 
	\end{itemize}

	\subsection{Erreurs}
	\begin{itemize}
		\item L'action empêche le bon déroulement de la simulation \warning
	\end{itemize}

	\subsection{Objet(s)}
		\allobjs

\section{enreg-config}
	\textbf{Qui:} \urt

	\subsection{Description}
	\begin{itemize}
		\item Collecte l'ensemble des paramètres de la configuration
		\item Sauvegarde la configuration
	\end{itemize}

	\subsection{Entrées}
		Nom de la simulation

	\subsection{Retours}
	\begin{itemize}
		\item Confirmation
		\item Configuration ajoutée à la base des configurations
	\end{itemize}

	\subsection{Erreurs}
	\begin{itemize}
		\item Paramètre des objets incomplets (seuls les objets correctement décrits sont sauvegardés)
	\end{itemize}

	\subsection{Objet(s)}
		\allobjs

\section{enreg-sous-config}
	Voir enreg-config

\section{charger-config}
	\textbf{Qui:} \urt

	\subsection{Description}
	\begin{itemize}
		\item Affiche la liste des configurations sauvegardées
		\item Charge la configuration sélectionnée
		\item Restaure les objets de la configuration
	\end{itemize}

	\subsection{Entrées}
		Nom de la configuration

	\subsection{Retours}
	\begin{itemize}
		\item Affiche l'interface de configuration
	\end{itemize}

	\subsection{Erreurs}
	\begin{itemize}
		\item Configuration inconnue \fatal
		\item Configuration incomplète \warning
	\end{itemize}

	\subsection{Objet(s)}
		\allobjs

\section{select-journal}
	\textbf{Qui:} \urt

	\subsection{Description}
	\begin{itemize}
		\item Sélection un journal des données d'exploitation
		\item Affiche les données d'exploitation les plus récentes
		\item Affiche les statistiques calculées d'après ces données d'exploitation
	\end{itemize}

	\subsection{Entrées}
		Nom du journal, contexte (configuration ou simulation)

	\subsection{Retours}
	\begin{itemize}
		\item Données d'exploitation récentes
		\item Statistiques
	\end{itemize}

	\subsection{Erreurs}
	\begin{itemize}
		\item Journal corrompu \fatal
	\end{itemize}

	\subsection{Objet(s)}
		\logs

\section{redem-sgbag}
	\textbf{Qui:} Tous les utilisateurs

	\subsection{Description}
	\begin{itemize}
		\item Sauvegarde le contexte de l'application
		\item Termine la session de travail de l'utilisateur
		\item Quitte l'application
		\item Démarre à nouveau l'application
	\end{itemize}

	\subsection{Entrées}
		sessid
		
	\subsection{Retours}
	\begin{itemize}
		\item Quitte et redémarre l'application
	\end{itemize}

	\subsection{Objet(s)}
		\u-Aéroport, \logs, \event

\section{}
	\textbf{Qui:} Tous les utilisateurs

	\subsection{Description}
	\begin{itemize}
		\item 
	\end{itemize}

	\subsection{Entrées}

	\subsection{Retours}
	\begin{itemize}
		\item 
	\end{itemize}

	\subsection{Erreurs}
	\begin{itemize}
		\item 
	\end{itemize}

	\subsection{Objet(s)}

\section{nvl-demande-interv}
	\textbf{Qui:} \us

	\subsection{Description}
	\begin{itemize}
		\item Enregistre la nouvelle demande d'intervention
		\item Transmet la demande d'intervention au responsable technique et au technicien informatique
		\item Enregistre la demande dans les données d'exploitation
	\end{itemize}

	\subsection{Entrées}
		Demande d'intervention, incident

	\subsection{Retours}
	\begin{itemize}
		\item Confirmation à l'utilisateur
		\item Création de demande d'intervention affichées aux utilisateurs concernés
	\end{itemize}

	\subsection{Erreurs}
	\begin{itemize}
		\item Données en entrée incomplètes \warning
	\end{itemize}

	\subsection{Objet(s)}
		\logs, \event

%%%  tpl %%%
%\section{}
%	\textbf{Qui:} Tous les utilisateurs
%
%	\subsection{Description}
%	\begin{itemize}
%		\item 
%	\end{itemize}
%
%	\subsection{Entrées}
%
%	\subsection{Retours}
%	\begin{itemize}
%		\item 
%	\end{itemize}
%
%	\subsection{Erreurs}
%	\begin{itemize}
%		\item 
%	\end{itemize}
%
%	\subsection{Objet(s)}
%%%
