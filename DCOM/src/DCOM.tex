% Shortcuts !
\newcommand\urt{Responsable technique}
\newcommand\us{Superviseur}
\newcommand\fatal{(Erreur fatale)}
\newcommand\warning{(message d'avertissement)}
\newcommand\event{u-Événement}
\newcommand\syslog{u-Données d'exploitation}
\newcommand\allobjs{u-Aéroport, u-Avion, u-Terminal, u-Hall, u-Chemin de connexion entre les halls, u-Gichet d'enregistrement,
u-Bagage, u-Chariot, u-Rail, u-Zone de contrôle de sécurité automatique, u-Zone de contrôle de sécurité
manuelle, u-Portique, u-Voie de garage, u-Zone de chargement des batteries, u-Zone embarquement, u-Tapis roulant,
u-Train, u-Wagonnet, u-Container, u-Plateau élévateur, u-Zone de déchargement, u-Tobogan, u-Zone de retrait des bagages,
u-Carrousel, u-Zone de maintenance, u-Aiguillage, u-Chemin de roulement, u-Événement, u-Données d'exploitation}
\newcommand\circobjs{u-Zone*, u-Bagage, u-Chariot, u-Voie de garage, u-Tapis roulant, u-Train, u-Rail, u-Wagonnet, u-Container,
u-Plateau Élévateur, u-Tobogan, u-Carrousel, u-Aiguillage, u-Chemin de roulement}
\newcommand\autobjs{u-Guichet d'enregistrement, u-Chariot, u-Portique, u-Tapis roulant, u-Train, u-Plateau élévateur, u-Tobogan,
u-Carrousel, u-Chemin de roulement, \event}

% Content...
\section{DAU(1) Accéder au système de gestion du SGBag}
	\textbf{Qui:} \urt, \us

	\subsection{Description}
	\begin{itemize}
		\item Authentification de l'utilisateur au système SGBag
		\item Déclenche l'évaluation des droits de l'utilisateurs
		\item Charge l'environnement de travail de l'utilisateur
	\end{itemize}

	\subsection{Entrées}
		login, password

	\subsection{Retours}
	\begin{itemize}
		\item confirmation de l'authentification
		\item identifiant de session de travail (sessid)
	\end{itemize}

	\subsection{Erreurs}
	\begin{itemize}
		\item cpoule (login, password) incorrect, inexistant dans le
		système ou ne disposant pas de droits suffisants \fatal
	\end{itemize}

	\subsection{Objet(s)}
		u-Aéroport, \event, \syslog

\section{DAU(2) Effectuer la simulation}
	\textbf{Qui:} \urt, \us

	\subsection{Description}
	\begin{itemize}
		\item Enregistre la configuration de la simulation
		\item Vérifie la validité de la simulation
		\item Effectue la simulation :
			\begin{itemize}
				\item Mise à jour des paramètre de simulation par l'utilisateur
				\item Mise à jour de l'interface utilisateur,
				\item Ajout de données dans l'interface de visualisation de la simulation
			\end{itemize}
	\end{itemize}

	\subsection{Entrées}
		Paramètres de la simulation

	\subsection{Retours}
	\begin{itemize}
		\item Activation de la simulation
		\item Mise à jour de l'interface de visualisation (déplacement des chariots, etc)
	\end{itemize}

	\subsection{Erreurs}
	\begin{itemize}
		\item Paramètres invalides \fatal
	\end{itemize}

	\subsection{Objet(s)}
		\allobjs

\section{DAU(12) Effectuer la configuration}
	\textbf{Qui:} \urt, \us

	\subsection{Description}
	\begin{itemize}
		\item Configuration du circuit
		\item Test de validité de la configuration
		\item Mise à jour de la configuration
		\item Sauvegarde de la configuration
		\item Test de réponse des éléments du système
		\item Enregistrement des modifications dans les données d'exploitation
	\end{itemize}

	\subsection{Entrées}
		Configuration du circuit

	\subsection{Retours}
	\begin{itemize}
		\item Mise à jour de l'interface
		\item Confirmation des tests
		\item État de la configuration
	\end{itemize}

	\subsection{Erreurs}
	\begin{itemize}
		\item Paramètres invalides \fatal
	\end{itemize}

	\subsection{Objet(s)}
		\allobjs

\section{DAU(17) Réagir en cas d'urgence}
	\textbf{Qui:} \urt, \us

	\subsection{Description}
	\begin{itemize}
		\item Enregistrement de l'alerte
		\item Demande d'intervention aux service concerné
		\item Arrêt des systèmes défaillants
		\item Mise en place de l'itinéraire de secours
		\item Enregistrement les actions dans les données d'exploitation
	\end{itemize}

	\subsection{Entrées}
		Description de l'urgence, personnel à contacter

	\subsection{Retours}
	\begin{itemize}
		\item Confirmation de la demande d'intervention (avec données de suivi)
		\item Mise à jour de l'état du circuit sur l'interface
	\end{itemize}

	\subsection{Objet(s)}
		\event, \syslog

\section{selec-journal}
	\textbf{Qui:} \urt

	\subsection{Description}
	\begin{itemize}
		\item Sélection du journal a étudier
		\item Chargement du journal
		\item affichage du journal
	\end{itemize}

	\subsection{Entrées}
		Idenfiant du journal

	\subsection{Retours}
	\begin{itemize}
		\item Contenu du journal
	\end{itemize}

	\subsection{Erreurs}
	\begin{itemize}
		\item Mauvais identifiant du journal \fatal
	\end{itemize}

	\subsection{Objet(s)}
		\syslog

\section{DAU(18) Analyser le document sélectionné}
	\textbf{Qui:} \urt

	\subsection{Description}
	\begin{itemize}
		\item Tri/Filtre des entrées
		\item Calcul des statistiques
		\item Affichage des résultats
	\end{itemize}

	\subsection{Entrées}
		Ordres de tri, Ordres de recherche, Ordres de calcul

	\subsection{Retours}
	\begin{itemize}
		\item Données d'exploitation triées
		\item Données d'exploitation filtrées
		\item Statistiques
	\end{itemize}

	\subsection{Erreurs}
	\begin{itemize}
		\item Mauvais paramètre \warning
	\end{itemize}

	\subsection{Objet(s)}
		\syslog

