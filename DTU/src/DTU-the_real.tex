\begin{description}
\item[U1) Technicien de maintenance] Le technicien de maintenance est appelé en
cas de problème mécanique ou électronique. Dans ce cas, tout ou partie du
système sera temporairement à l'arrêt. Le technicien devras avoir accès à des
commandes simples permettant le test des fonctionnalités en causes. Dans un
second temps, le technicien devras pouvoir remplir une fiche électronique
d'intervention.
\item[U2) Conducteur] Le conducteur est chargé de gérer et conduire un train de
bagages. Il a accès à une fonction permettant d'interrompre le chargement d'un
wagon de bagages quand il est plein, et de le reprendre lorsqu'il à déplacé le
train pour remplir le wagon suivant.
\item[U6) Responsable de maintenance] Le responsable de maintenance reçoit les
demandes d'entretiens et est chargé de les distribuer parmi ses équipes. Tout
comme les techniciens de maintenance, il à accès aux commandes de tests et
soumission de fiches électroniques. Mais il à également un accès privilegié sur
la base de données des fiches électroniques permettant une validation et une
gestion simple.
\item[U7) Superviseur] Le superviseur gère tout un secteur du système de
bagagerie, il à donc accès à toutes les informations disponibles sur le système
en temps réel. Il doit également pouvoir stopper tout ou partie de système afin
de prévenir d'une panne, ou de stopper un éléments disfonctionnel. Il doit
pouvoir avertir le service informatique ou le service maintenance en cas de panne.
\item[U9) Responsable technique] Le responsable technique est sous les ordres du
responsable des bagages. Il à acces à toutes les fonctions du superviseur.
\item[U10) Responsable du service bagages] Le responsable du service des bagages
à pour tâche de diriger les superviseurs. Il pilote l'ensemble des activités du
services et a donc à sa disposition les outils de contrôle et de maintenance
auxquels le superviseur (U7) a accès.
\item[U11) Technicien informatique] Le technicien informatique assure le
support et la maintenance des outils informatiques de l'aéroport. Il doit
pouvoir avertir le service entretien en cas de panne materielle. Il doit avoir
accès à l'ensemble des outils informatique utilisés par les autres profils.
\item[U12) Responsable informatique] Le responsable informatique supervise
l'équipe des techniciens informatique. Dans le contexte du service bagages, ses
tâches sont les mêmes que celles du technicien informatique.
\end{description}