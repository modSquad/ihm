U1) Technicien de maintenance
Le technicien de maintenance est appelé en cas de problème mécanique ou électronique.
Dans ce cas, tout ou partie du système sera temporairement à l'arrêt.
Le technicien devras avoir accès à des commandes simples permettant le test des fonctionnalités en causes.
Dans un second temps, le technicien devras pouvoir remplir une fiche électronique d'intervention.

U6) Responsable de maintenance
Le responsable de maintenant reçoit les demande d'entretiens et est chargé de les distribué parmi ses équipes.
Tout comme les techniciens de maintenance, il à accès aux commandes de tests et soumission de fiches électroniques.
Mais il à également un acces privilegié sur la base de données des fiches electroniques permettant une validation et une gestion simple.

U7) Superviseur
Le superviseur gère tout un secteur du système de bagagerie, il à donc accès à toutes les informations disponibles sur le système en temps réel.
Il doit également pouvoir stopper tout ou partie de système afin de prévenir d'une panne, ou de stopper un éléments disfonctionnel.
Il doit pouvoir avertir le service informatique ou le service maintenance en cas de panne.

U9) Responsable technique
Le responsable technique est sous les ordres du responsable des bagages.
Il à acces à toutes les fonctions du superviseur.