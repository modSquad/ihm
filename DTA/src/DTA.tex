\part{Description des tâches d'activités}

\dta
{2}{8}{Conducteur}
{ALT}
{
\begin{description}
	\item [T2.8.1)] Gérer l'embarquement des bagages : REP
	\begin{description}
		\item [T2.8.1.1)] Vérification visuelle du remplissage des conteneurs [\textsl{manuel}]
		\item [T2.8.1.2)] Embarquement des bagages [\textsl{manuel}]
		\item [T2.8.1.3)] Signaler au SGBag l'avancement des opérations. [\textsl{système}]
	\end{description}
	\item [T2.8.2)] Gérer le débarquement des bagages
		\begin{description}
			\item [T2.8.2.1)] Débarquement des bagages [\textsl{manuel}]
			\item [T2.8.2.2)] Signaler au SGBag la fin des opérations. [\textsl{système}]
		\end{description}
\end{description}
}
{1}{6}{Technicien de Maintenance}
{SEQ}
{
\begin{description}
	\item[T1.6.1] Réparation\textbf{SEQ}
	\begin{description}
		\item[T1.6.1.1] Résoudre une demande d'intervention\textbf{SEQ}
		\begin{description}
			\item[T1.6.1.1.1] Receptionner une demande d'intervention [système]
			\item[T1.6.1.1.2] Intervenir [manuelle]
			\item[T1.6.1.1.3] Verifier l'efficacité de l'intervention [système]
			\item[T1.6.1.1.4] Remplir la fiche électronique d'intervention [système]
		\end{description}
	\end{description}
\end{description}
}

\hspace{1cm}

\dta
{3}{7}{Contrôleur}
{ALT, SEQ, REP}
{
\begin{description}
	\item [T3.7.1)] Recoit les alertes de bagages suspects [\textsl{système}]
	\item [T3.7.2)] Effectuer un contrôle visuel à distance : ALT
	\begin{description}
		\item [T3.7.3.1)] Orienter le chariot vers une zone de contrôle manuel [\textsl{système}]
		\item [T3.7.3.2)] Orienter vers le chemin \og normal\fg) [\textsl{système}]
	\end{description}
	\item [T3.7.5)] Enregistrer le passage du bagages ainsi que la décision associée [\textsl{système}]
\end{description}
}

\hspace{1cm}

\dta
{4}{3}{Opérateur de guichet}
{REP, SEQ}
{
\begin{description}
	\item [T4.3.1)] Acceuillir le client [\textsl{manuel}]
	\item [T4.3.2)] Utiliser le système de pesée
	\begin{description}
		\item [T4.3.2.1)] Lancer la pesée [\textsl{système}]
		\item [T4.3.2.1)] Contrôler la pesée [\textsl{système}]
	\end{description}
	\item [T4.3.3)] Enregistrer le bagage [\textsl{système}]
	\begin{description}
		\item [T4.3.3.1] Imprimer l'étiquette \textsl{RFID} [\textsl{système}]
		\item [T4.3.3.1] Attacher l'étiquette au bagage [\textsl{système}]
	\end{description}
	\item [T4.3.4)] Injecter le(s) bagages(s) dans le \textsl{SGBag}  [\textsl{système}]
\end{description}
}

\hspace{1cm}

\dta
{5}{3}{Voyageur (lésé ou pas)}
{SEQ}
{
\begin{description}
	\item [T5.3.1)] Enregistrer ses bagages auprès de l'opérateur de guichet [\textsl{système}]
	\item [T5.3.2)] Passer le contrôle de sécurité [\textsl{manuel}]
	\item [T5.3.3)] Embarquer [\textsl{manuel}]
	\item [T5.3.4)] Débarquer : ALT
	\begin{description}
		\item [T5.3.5.1)] Récupérer ses bagages sur le carrousel [\textsl{manuel}]
		\item [T5.3.6)] Faire appel au service des réclamations [\textsl{manuel}]
		\item [T5.3.6.1)] Suivre son dossier via INTERNET ou via le téléphone [\textsl{manuel}]
	\end{description}
	\item [T5.3.7)] Passer la douane et éventuellement le service d'immigration. [\textsl{manuel}]
\end{description}
}

\hspace{1cm}

\dta
{6}{5, 6}{Responsable de Maintenance}
{SEQ}
{
\begin{description}
	\item[T6.5.1] Répartir les demande d'intervention \textbf{SEQ}
	\begin{description}
		\item[T6.5.1.1] Receptionner les demandes d'intervention [système]
		\item[T6.5.1.2] Répartir les demandes d'intervention [système]
		\item[T6.5.1.3] Receptionner et valider les fiches électronique d'intervention [système]
	\end{description}
	\item[T6.6.1] Réparation\textbf{SEQ}
	\begin{description}
		\item[T6.6.1.1] Résoudre une demande d'intervention \textbf{SEQ}
		\begin{description}
			\item[T6.6.1.1.1] Receptionner d'une demande d'intervention [système]
			\item[T6.6.1.1.2] Intervenir [manuelle]
			\item[T6.6.1.1.3] Verifier l'éfficacité de intervention [système]
			\item[T6.6.1.1.4] Remplir la fiche électronique d'intervention [système]
		\end{description}
	\end{description}
\end{description}
}

\hspace{1cm}

\dta
{7}{1, 6}{Superviseur}
{SEQ}
{
\begin{description}
	\item[T.7.1.1] Prend en charge la simulation sur un terminal.
	\item[T.7.6.1] Surveiller l'Équipement \textbf{SEQ}
	\begin{description}
		\item[T.7.6.1.1] Vérifier le bon état de l'Équipement [système]
		\item[T.7.6.1.2] Si (problème) : arrêter les éléments concernés, deposer une demande d'intervention au service concerné [système]
	\end{description}
\end{description}
}

\hspace{1cm}

\dta
{8}{3, 8}{Préposé aux réclamations}
{REP}
{
\begin{description}
	\item [T8.3.1)] Recevoir la description du problème depuis le service \og Réclamation\fg. [\textsl{système}]
	\item [T8.3.2)] Constitution du dossier [\textsl{système}] 
	\item [T8.3.3)] Fournir un numéro d'identification et un mot de passe [\textsl{système}]
	\item [T8.3.4)] Informer le client de l'avancement de son dossier [\textsl{manuel}]
	\item [T8.8.1)] Rechercher la trace d'un bagage égaré
	\begin{description}
		\item [T8.8.1.1)] Visualiser le trajet du bagage [\textsl{système}]
		\item [T8.8.1.2)] Détecter l'erreur [\textsl{système}]
	\end{description}
\end{description}
}

\hspace{1cm}

\dta
{10}{5}{Responsable du service des bagages}{~}
{
\begin{description}
	\item [T10.5.1)] Diriger les superviseurs
\end{description}
}

\dta
{11}{1}{Technicien informatique}{~}
{
\begin{description}
\item [T11.1.1)] Faire remonter des suggestions d’amélioration OPT
	\item [T11.9.1)] Intervenir sur un problème informatique SEQ
	\begin{description}
		\item [T11.9.1.1)] Receptionner d’une demande d’intervention [système]
		\item [T11.9.1.2)] Intervenir [manuelle]
		\item [T11.9.1.3)] Si (panne matérielle), faire une demande d’intervention par le service maintenance [système]
		\item [T11.9.1.4)] Si (intervention terminée) Remplir la fiche électronique d’intervention [système]
	\end{description}
	\item [T11.9.2 Consulter les traces d’exécution : SEQ 
	\begin{description}
		\item [T11.9.2.1)] Lors d’une intervention, oenser qu’il y a un besoin d’analyser les opéra-
			tions effectuées sur le système [manuelle] 
		\item [T11.9.2.2)] Accéder aux traces d’exécution [système]
	\end{description}
\end{description}
}

\dta
{12}{1, 9}{Responsable informatique}{~}
{
\begin{description}
	\item [T12.1.1)] Faire remonter des suggestions d’amélioration OPT
	\item [T12.1.1)] Intervenir sur un problème informatique (voir T1.11.1) 
	\item [T12.1.2)] Consulter les traces d’exécution (voir T1.11.2)
\end{description}
}

{9}{1, 6}{Superviseur}
{SEQ}
{
\begin{description}
	\item[T.9.1.1] Configurer le système avant la mise en fonctionnement.
	\item[T.9.1.2] Prend en charge la simulation sur l'ensemble de l'aéroport.
	\item[T.9.1.3] Examiner les traces pour analyse de performances, pour des statistiques de pannes, et pour initialiser le système dans l'état où il était avant un arrêt.
	\item[T.9.6.1] Surveiller l'Équipement
	\begin{description}
		\item[T.9.6.1.1] Vérifier le bon état de l'Équipement [système]
		\item[T.9.6.1.2] Si (problème), arrêter les éléments concernés, deposer un
		demande d'intervention au service concerné, ou contacter le superviseur concerné [système]
	\end{description}
\end{description}
}
