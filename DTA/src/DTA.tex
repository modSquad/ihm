\part{Description des tâches d'activités}

\dta
{3}{X}{Contrôleur}
{REP}
{
\begin{description}
	\item [T3.X.1)] Utiliser le service d'alert des bagages suspects
	\item [T3.X.2)] Effectuer un contrôle visuel à distance
	\item [T3.X.3)] Orienter le chariot (vers une zone de contrôle manuel, vers la zone de traitement des bagages douteux ou vers le chemin \og normal\fg)
	\item [T3.X.4)] Effectuer un contrôle manuel
	\item [T3.X.5)] Enregistrer le passage du bagages ainsi que la décision associée	
\end{description}
}

\dta
{4}{X}{Opérateur de guichet}
{REP, SEQ}
{
\begin{description}
	\item [T4.X.1)] Acceuillir le client
	\item [T4.X.2)] Utiliser le système de pesée
		\subitem [T4.X.2.1)] Lancer la pesée
		\subitem [T4.X.2.1)] Contrôler la pesée

	\item [T4.X.3)] Enregistrer le bagage
		\subitem [T4.X.3.1] Imprimer l'étiquette \textsl{RFID}
		\subitem [T4.X.3.1] Attacher l'étiquette au bagage
		
	\item [T4.X.4)] Injecter le(s) bagages(s) dans le \textsl{SGBag} 
\end{description}
}i

\dta
{5}{X}{Voyageur}
{
\begin{description}
	\item [T5.X.1)] Enregistrer ses bagages
	\item [T5.X.2)] Embarquer
	\item [T5.X.3)] Passer le contrôle de sécurité
	\item [T5.X.4)] Débarquer
	\item [T5.X.5)] Récupérer ses bagages sur le carrousel
	\item [T5.X.6)] Faire appel au service des réclamations
		\subitem [T5.X.6.1)] Suivre son dossier via INTERNET ou via le téléphone
	\item [T5.X.7)] Passer la douane et éventuellement le service d'immigration.
\end{description}
}

\dta
{8}{X}{Préposé aux réclamations}
{
\begin{description}
	\item [T8.X.1)] Rechercher la trace d'un bagage égaré
		\subitem [T8.X.1.2] Visualiser le trajet du bagage
	\item [T8.X.2)] Traiter un dossier de voyageur
		\subitem [T8.X.2.1)] Ouvrir un dossier
		\subitem [T8.X.2.1)] Consulter un dossier
		\subitem [T8.X.2.1)] Manier le système d'identification un dossier
			\subsubitem [T8.X.2.1)] Fournir un numéro d'identification et un mot de passe
	\item [T8.X.3)] Informer le client de l'avancement de son dossier
\end{description}
}
