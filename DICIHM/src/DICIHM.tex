\section{Chartre graphique \& Guide de style}
La chartre graphique et le guide de style de utilisé pour l'\textsl{IHM} seront ceux de l'aéroport. On y retrouvera donc leur logo, leur thème de couleur, \textsl{etc.}

\section{Métaphore}
La métaphore choisie est celle de \og l'aéroport\fg. Le choix de cette métaphore est trivial mais il a l'avantage de regrouper les notions de \og transport, bagages, voyageurs, SGBag, \textsl{etc.}\fg.

\section{Tableau des principaux objets utilisateurs}
% Ne pas afficher les D_JSPU si la fonction associée n'existe pas
\providecommand{\DJSPU}[1]{}

\begin {tabular} {| l | l | l |}
\hline
~
&\begin{sideways}U7 Superviseur\end{sideways}
&\begin{sideways}U9 Responsable technique~\end{sideways}\\
\hline
u-SGBag &x &x \\
\hline
u-Configuration &x &x \\
\hline
u-Simulation &x &x \\
\hline
u-Aeroport & &x \\
\hline
u-Avion  &x &x \\
\hline
u-Terminal  &x &x \\
\hline
u-Chemin de connection entre les halls &x &x \\
\hline
u-Hall  &x &x \\
\hline
u-Guichet d'enregistrement &x &x \\
\hline
u-Bagage  &x &x \\
\hline
u-Chariot  &x &x \\
\hline
u-Rail &x &x \\
\hline
u-Zone contrôle de sécurité Automatique &x &x \\
\hline
u-Portique &x &x \\
\hline
u-Zone de contrôle de sécurité Manuelle &x &x \\
\hline
u-Voie de garage &x &x \\
\hline
u-Zone de chargement des batteries  &x &x \\
\hline
u-Zone embarquement  &x &x \\
\hline
u-Tapis roulant  &x &x \\
\hline
u-Train  &x &x \\
\hline
u-Wagonnets  &x &x \\
\hline
u-Containers  &x &x \\
\hline
u-Plateau elevateur  &x &x \\
\hline
u-Zone déchargement  &x &x \\
\hline
u-Tobogan  &x &x \\
\hline
u-Zone retrait des bagages  &x &x \\
\hline
u-Carrousel  &x &x \\
\hline
u-Zone de maintenance &x &x \\
\hline
u-Aiguillage &x &x \\
\hline
u-Chemin de roulement &x &x \\
\hline
u-\'{E}vénement &x &x \\
\hline
u-Données d'exploitations &x &x \\
\hline
\end {tabular}
